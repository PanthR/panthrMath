\documentclass{article}
\usepackage{listings}
\usepackage[vmargin=1in,hmargin=1in]{geometry}
\title{PanthRBase Documentation}
\author{PanthR Team}
\begin{document}
\maketitle

\tableofcontents

  \section{panthrMath}
    \subsection*{C}
    \addcontentsline{toc}{subsection}{C}
    Frequently used constants.  See \texttt{module:C}


    \subsection*{Polynomial(coefs)}
    \addcontentsline{toc}{subsection}{Polynomial(coefs)}
    Polynomial Class


    \subsection*{Polynomial.new(coefs)}
    \addcontentsline{toc}{subsection}{Polynomial.new(coefs)}
    TODO


    \subsection*{Rational(num, denom)}
    \addcontentsline{toc}{subsection}{Rational(num, denom)}
    Rational Class


    \subsection*{Rational.new(num, denom)}
    \addcontentsline{toc}{subsection}{Rational.new(num, denom)}
    TODO


    \subsection*{basicfunc}
    \addcontentsline{toc}{subsection}{basicfunc}
    A collection of basic and special functions. See \texttt{module:basicFunc}.


    \subsection*{distributions}
    \addcontentsline{toc}{subsection}{distributions}
    Various probability distributions. See \texttt{module:distributions}.


    \subsection*{Polynomial.prototype.evalAt(x)}
    \addcontentsline{toc}{subsection}{Polynomial.prototype.evalAt(x)}
    TODO


    \subsection*{Rational.prototype.evalAt(x)}
    \addcontentsline{toc}{subsection}{Rational.prototype.evalAt(x)}
    TODO


  \section{basicFunc}
    \subsection*{bd0(x, y)}
    \addcontentsline{toc}{subsection}{bd0(x, y)}
    Computes the expression $$bd0(x, y) = x \ln(x/y) + y - x$$


Based on:  \emph{Fast and Accurate Computation of Binomial Probabilities},
by Catherine Loader, 2000


    \subsection*{beta(a, b)}
    \addcontentsline{toc}{subsection}{beta(a, b)}
    Computes the beta function
$$B(a,b) = \int\_0^1 x^{a-1} (1-x)^{b-1} dx$$ where $a\&gt;0$ and $b\&gt;0$.


See also: \texttt{lbeta}


Based on: \emph{Algorithm 708: Significant Digit Computation of the Incomplete Beta Function
Ratios}, by DiDonato and Morris, 1992


    \subsection*{bratio(a, b, x, lowerTail, logp)}
    \addcontentsline{toc}{subsection}{bratio(a, b, x, lowerTail, logp)}
    Returns the lower tail of the incomplete beta function
$$I\_x(a, b)=G(a,b)\int\_0^xt^{a-1}(1-t)^{b-1}dt$$
where $B(a,b)=1/G(a,b) = \Gamma(a)\Gamma(b)/\Gamma(a+b)$ is the
usual beta function.


If \texttt{lowerTail} is \texttt{false}, returns the upper tail instead.
If \texttt{logp} is \texttt{true}, returns the logarithm of the tail.
Expects $a\&gt;0$, $b\&gt;0$, and $0 \leq x \leq 1$.


Based on: \emph{Algorithm 708: Significant Digit Computation of the Incomplete Beta Function
Ratios}, by DiDonato and Morris, 1992


    \subsection*{erf(x)}
    \addcontentsline{toc}{subsection}{erf(x)}
    Implementation for erf, the real error function:
$$\textrm{erf}(x) = \frac{2}{\sqrt{\pi}}\int\_0^x e^{-t^2} dt$$
Based on: \emph{Rational Chebyshev Approximations for the Error
Function}, by W. J. Cody, 1969


    \subsection*{erfc(x)}
    \addcontentsline{toc}{subsection}{erfc(x)}
    Implementation for erfc, the complement of the real error function:
$$\textrm{erfc}(x) = \frac{2}{\sqrt{\pi}}\int\_x^\infty e^{-t^2} dt$$
Based on: \emph{Rational Chebyshev Approximations for the Error
Function}, by W. J. Cody, 1969


    \subsection*{expm1(x)}
    \addcontentsline{toc}{subsection}{expm1(x)}
    Computes $$\textrm{expm1}(x) = e^x - 1$$ with reasonable precision near $x = 0$.


Based on:  \emph{Computation of the Incomplete Gamma Function Ratios
and their Inverse}, by DiDonato and Morris, 1986


    \subsection*{gam1(x)}
    \addcontentsline{toc}{subsection}{gam1(x)}
    Computes $$H(x)=\frac{1}{\Gamma(x+1)} - 1$$ with reasonable precision even near
the zeroes ($x = 0, 1$).


Based on:  \emph{Computation of the Incomplete Gamma Function Ratios
and their Inverse}, by DiDonato and Morris, 1986


    \subsection*{gaminv(a)(p, lower)}
    \addcontentsline{toc}{subsection}{gaminv(a)(p, lower)}
    Returns the inverse function
of the incomplete gamma ratio (see: \texttt{gratio}).


\texttt{a} is the shape parameter and \texttt{p} is the desired tail probability.
\texttt{lower} is a boolean which determines whether \texttt{p} is a left-tail
probability or not (defaults to \texttt{true}).


See also: \texttt{qgamma}


Based on:  \emph{Computation of the Incomplete Gamma Function Ratios
and their Inverse}, by DiDonato and Morris, 1986


    \subsection*{gamma(x)}
    \addcontentsline{toc}{subsection}{gamma(x)}
    Computes $$\Gamma(x) = \int\_0^\infty u^{x-1} e^{-u} du$$


Inspired by code in the Gnu Scientific Library


Also see: \emph{A Precision Approximation of the Gamma Function}, by
Lanczos, 1964


    \subsection*{gratR(a, x, logr)}
    \addcontentsline{toc}{subsection}{gratR(a, x, logr)}
    Computes the scaled complement of the incomplete gamma ratio function
(see: \texttt{gratio}): $$\textrm{grat}\_r(a,x,r) =  \frac{Q(a,x)}{r}$$
where
              $Q(a,x)$ is the upper-tail incomplete gamma function
(see: \texttt{gratioc}) and  $r = \frac{e^{-x} x^a}{\Gamma(a)}$


Meant to be used internally.


Based on:   \emph{Computation of the Incomplete Gamma Function Ratios
and their Inverse}, by DiDonato and Morris, 1986


    \subsection*{gratio(a)(x)}
    \addcontentsline{toc}{subsection}{gratio(a)(x)}
    Returns the incomplete gamma ratio:
$$P(a,x) = \frac{1}{\Gamma(a)} \int\_0^x e^{-t} t^{a-1} dt$$


\texttt{a} is the shape parameter and $0 \&lt; x \&lt; \infty$.


See also: \texttt{pgamma}


Based on:  \emph{Computation of the Incomplete Gamma Function Ratios
and their Inverse}, by DiDonato and Morris, 1986


    \subsection*{gratioc(a)(x)}
    \addcontentsline{toc}{subsection}{gratioc(a)(x)}
    Returns the complement of the incomplete gamma ratio:
$$Q(a,x) = \frac{1}{\Gamma(a)} \int\_x^\infty e^{-t} t^{a-1} dt$$


\texttt{a} is the shape parameter and $0 \&lt; x \&lt; \infty$.


See also: \texttt{pgamma}


Based on:  \emph{Computation of the Incomplete Gamma Function Ratios
and their Inverse}, by DiDonato and Morris, 1986


    \subsection*{lbeta(a, b)}
    \addcontentsline{toc}{subsection}{lbeta(a, b)}
    Computes the logarithm of the beta function.
See: \texttt{beta}


Based on: \emph{Algorithm 708: Significant Digit Computation of the Incomplete Beta Function
Ratios}, by DiDonato and Morris, 1992


    \subsection*{lgamma(x)}
    \addcontentsline{toc}{subsection}{lgamma(x)}
    Computes the logarithm of the gamma function.  See: \texttt{gamma}


Inspired by code in the Gnu Scientific Library


Also see: \emph{A Precision Approximation of the Gamma Function}, by
Lanczos, 1964


    \subsection*{log1p(x)}
    \addcontentsline{toc}{subsection}{log1p(x)}
    Computes $$\textrm{log1p}(x) = \ln(1+x)$$ with reasonable precision near $x = 0$.


Based on:  \emph{Computation of the Incomplete Gamma Function Ratios
and their Inverse}, by DiDonato and Morris, 1986


    \subsection*{lpoisson(lambda)(x)}
    \addcontentsline{toc}{subsection}{lpoisson(lambda)(x)}
    Computes the logarithm of the Poisson pdf
$$\log\left(\frac{\lambda^x}{\Gamma(x+1)}e^{-\lambda}\right)
= -\lambda + x \ln(\lambda) - \ln(\Gamma(x+1))$$
where $x\&gt;0$ and $\lambda\&gt;0$.


Based on:  \emph{Fast and Accurate Computation of Binomial Probabilities},
by Catherine Loader, 2000


    \subsection*{phi(x)}
    \addcontentsline{toc}{subsection}{phi(x)}
    Computes $$\phi(x) = x - 1 - \ln(x)$$ with reasonable precision
near $x = 1$.


Based on:  \emph{Computation of the Incomplete Gamma Function Ratios
and their Inverse}, by DiDonato and Morris, 1986


    \subsection*{stirlerr(n)}
    \addcontentsline{toc}{subsection}{stirlerr(n)}
    Computes the error term in Stirling's approximation
$$\textrm{stirlerr}(n) = \ln(\Gamma(n+1)) - n \ln(n) + n - \frac{1}{2} \ln\left(2\pi n\right)$$
where $n\&gt;0$ (\texttt{n} does not need to be an integer).


Based on:  \emph{Fast and Accurate Computation of Binomial Probabilities},
by Catherine Loader, 2000


  \section{C}
    \subsection*{C}
    \addcontentsline{toc}{subsection}{C}
    Constants module, provides frozen values for some frequently-used constants.


    \subsection*{eulerGamma}
    \addcontentsline{toc}{subsection}{eulerGamma}
    The Euler-Mascheroni constant
\[\gamma = \lim \_ {n \to \infty} \left(-\ln(n) + \sum \_ {k=1}^n \frac{1}{k}\right) = 0.57721566 \ldots\]


    \subsection*{log2pi}
    \addcontentsline{toc}{subsection}{log2pi}
    $\ln(2\pi)$


    \subsection*{sqrt2pi}
    \addcontentsline{toc}{subsection}{sqrt2pi}
    $\sqrt{2\pi}$


    \subsection*{twopi}
    \addcontentsline{toc}{subsection}{twopi}
    $2\pi$


  \section{distributions}
    \subsection*{beta}
    \addcontentsline{toc}{subsection}{beta}
    The beta distribution. See \texttt{module:beta}.


    \subsection*{binom}
    \addcontentsline{toc}{subsection}{binom}
    The binomial distribution. See \texttt{module:binomial}.


    \subsection*{finite}
    \addcontentsline{toc}{subsection}{finite}
    Constructor for finite distributions. See \texttt{module:finite}.


    \subsection*{gamma}
    \addcontentsline{toc}{subsection}{gamma}
    The gamma distribution. See \texttt{module:gamma}.


    \subsection*{normal}
    \addcontentsline{toc}{subsection}{normal}
    The normal distribution. See \texttt{module:normal}.


    \subsection*{poisson}
    \addcontentsline{toc}{subsection}{poisson}
    The Poisson distribution. See \texttt{module:poisson}.


    \subsection*{t}
    \addcontentsline{toc}{subsection}{t}
    The t distribution. See \texttt{module:t}.


    \subsection*{uniform}
    \addcontentsline{toc}{subsection}{uniform}
    The uniform distribution. See \texttt{module:uniform}.


  \section{beta}
    \subsection*{betadistr(a, b)}
    \addcontentsline{toc}{subsection}{betadistr(a, b)}
    Returns an object representing a beta distribution, with properties \texttt{d}, \texttt{p}, \texttt{q}, \texttt{r}.


\begin{lstlisting}
\texttt{betadistr(a, b).d(x, logp)            // same as dbeta(a, b, logp)(x)
betadistr(a, b).p(x, lowerTail, logp) // same as pbeta(a, b, lowerTail, logp)(x)
betadistr(a, b).q(x, lowerTail, logp) // same as qbeta(a, b, lowerTail, logp)(x)
betadistr(a, b).r(n)                  // same as rbeta(a, b)(n)}\end{lstlisting}

    \subsection*{dbeta(a, b, logp)(x)}
    \addcontentsline{toc}{subsection}{dbeta(a, b, logp)(x)}
    Evaluates the Beta density function at \texttt{x}:
$$\textrm{dbeta}(a,b)(x) = \frac{\Gamma(a+b)}{\Gamma(a)\Gamma(b)}x^{(a-1)}(1-x)^{(b-1)}$$


Expects $a \&gt; 0$, $b \&gt; 0$, and $0 \leq x \leq 1$.


\texttt{logp} defaults to \texttt{false}; if \texttt{logp} is \texttt{true}, returns the
logarithm of the result.


    \subsection*{pbeta(a, b, lowerTail, logp)(x)}
    \addcontentsline{toc}{subsection}{pbeta(a, b, lowerTail, logp)(x)}
    Evaluates the Beta cumulative distribution
function at \texttt{x} (lower tail probability):
$$\textrm{pbeta}(a, b)(x) = I\_x(a, b)=G(a,b)\int\_0^xt^{a-1}(1-t)^{b-1}dt$$
where $B(a,b)=1/G(a,b) = \Gamma(a)\Gamma(b)/\Gamma(a+b)$ is the
usual beta function.


\texttt{lowerTail} defaults to \texttt{true}; if \texttt{lowerTail} is \texttt{false}, returns
the upper tail probability instead.


\texttt{logp} defaults to \texttt{false}; if \texttt{logp} is \texttt{true}, returns the logarithm
of the result.


Expects $a\&gt;0$, $b\&gt;0$, and $0 \leq x \leq 1$.


Based on: \emph{Algorithm 708: Significant Digit Computation of the Incomplete Beta Function
Ratios}, by DiDonato and Morris, 1992


    \subsection*{qbeta(a, b, lowerTail, logp)(p)}
    \addcontentsline{toc}{subsection}{qbeta(a, b, lowerTail, logp)(p)}
    Evaluates the Beta quantile function (inverse cdf) at \texttt{p}:
$$\textrm{qbeta}(a, b)(p) = x \textrm{ such that } \textrm{prob}(X \leq x) = p$$
where $X$ is a random variable with the $\textrm{Beta}(a,b)$ distribution.


\texttt{lowerTail} defaults to \texttt{true}; if \texttt{lowerTail} is \texttt{false}, \texttt{p} is
interpreted as an upper tail probability (returns
$x$ such that $\textrm{prob}(X \&gt; x) = p)$.


\texttt{logp} defaults to \texttt{false}; if \texttt{logp} is \texttt{true}, interprets \texttt{p} as
the logarithm of the desired probability.


Expects $a\&gt;0$, $b\&gt;0$, and $0 \leq p \leq 1$.


    \subsection*{rbeta(a, b)}
    \addcontentsline{toc}{subsection}{rbeta(a, b)}
    Returns a random variate from the $\textrm{Beta}(a, b)$ distribution.


Expects $a\&gt;0$ and $b\&gt;0$.


Based on R's code; see: \emph{Generating beta variates with nonintegral shape parameters}, by
RCH Cheng, 1978


  \section{binomial}
    \subsection*{binom(size, p)}
    \addcontentsline{toc}{subsection}{binom(size, p)}
    Returns an object representing a binomial distribution, with properties \texttt{d}, \texttt{p}, \texttt{q}, \texttt{r}.


\begin{lstlisting}
\texttt{binom(size, p).d(x, logp)            // same as dbinom(size, p, logp)(x)
binom(size, p).p(x, lowerTail, logp) // same as pbinom(size, p, lowerTail, logp)(x)
binom(size, p).q(x, lowerTail, logp) // same as qbinom(size, p, lowerTail, logp)(x)
binom(size, p).r()                   // same as rbinom(size, p)()}\end{lstlisting}

    \subsection*{dbinom(size, p, logp)(x)}
    \addcontentsline{toc}{subsection}{dbinom(size, p, logp)(x)}
    Returns the Binomial probability at \texttt{x}:
$$\textrm{dbinom}(\textrm{size}, p)(x) = \binom{\textrm{size}}{p}p^{x}(1-p)^{(\textrm{size}-x)}$$
where \texttt{x} is an integer, $0 \leq x \leq \textrm{size}$.


\texttt{size} is a positive integer (number of trials) and $0 \leq p \leq 1$
(the probability of success on a single trial).


\texttt{logp} defaults to \texttt{false}; if \texttt{logp} is \texttt{true}, returns the
logarithm of the result.


Based on: \emph{Fast and Accurate Computation of Binomial Probabilities},
by Catherine Loader, 2000


    \subsection*{pbinom(size, p, lowerTail, logp)(x)}
    \addcontentsline{toc}{subsection}{pbinom(size, p, lowerTail, logp)(x)}
    Evaluates the Binomial cumulative distribution
function at \texttt{x} (lower tail probability):
$$\textrm{pbinom}(\textrm{size}, p)(x) = \sum\_{k \leq x} \binom{\textrm{size}}{p}p^{k}(1-p)^{(\textrm{size}-k)}$$


\texttt{size} is a positive integer (number of trials) and $0 \leq p \leq 1$
(the probability of success on a single trial).


\texttt{lowerTail} defaults to \texttt{true}; if \texttt{lowerTail} is \texttt{false}, returns
the upper tail probability instead:
$$\textrm{pbinom}(\textrm{size}, p, \textrm{false})(x) = \sum\_{k\&gt;x} \binom{\textrm{size}}{p}p^{k}(1-p)^{(\textrm{size}-k)}$$


\texttt{logp} defaults to \texttt{false}; if \texttt{logp} is \texttt{true}, returns the logarithm
of the result.


    \subsection*{qbinom(size, p, lowerTail, logp)(prob)}
    \addcontentsline{toc}{subsection}{qbinom(size, p, lowerTail, logp)(prob)}
    Returns the quantile corresponding to \texttt{prob}
for the $\textrm{Binomial}(size, p)$ distribution.
In general, for a discrete probability
distribution, the \emph{quantile} is defined as the smallest domain value
\texttt{x} such that $F(x) \geq prob$, where $F$ is the cumulative
distribution function.


\texttt{size} is a positive integer (number of trials) and $0 \leq p \leq 1$
(the probability of success on a single trial).


\texttt{lowerTail} defaults to \texttt{true}; if \texttt{lowerTail} is \texttt{false}, \texttt{prob} is
interpreted as an upper tail probability.


\texttt{logp} defaults to \texttt{false}; if \texttt{logp} is \texttt{true}, interprets \texttt{prob} as
the logarithm of the desired probability.


    \subsection*{rbinom(size, p)}
    \addcontentsline{toc}{subsection}{rbinom(size, p)}
    Returns a random variate from the $\textrm{Binomial}(size, p)$ distribution.


\texttt{size} is a positive integer (number of trials) and $0 \leq p \leq 1$
(the probability of success on a single trial).


  \section{finite}
    \subsection*{dfinite(o, logp)(x)}
    \addcontentsline{toc}{subsection}{dfinite(o, logp)(x)}
    Returns the probability at \texttt{x} for the finite distribution
represented by object \texttt{o}.


\texttt{logp} defaults to \texttt{false}; if \texttt{logp} is \texttt{true}, returns the
logarithm of the result.


    \subsection*{finite(o)}
    \addcontentsline{toc}{subsection}{finite(o)}
    Returns an object representing a finite distribution, with properties
\texttt{d}, \texttt{p}, \texttt{q}, \texttt{r}.


\begin{lstlisting}
\texttt{finite(o).d(x, logp)            // same as dfinite(o, logp)(x)
finite(o).p(x, lowerTail, logp) // same as pfinite(o, lowerTail, logp)(x)
finite(o).q(x, lowerTail, logp) // same as qfinite(o, lowerTail, logp)(x)
finite(o).r(n)                  // same as rfinite(o)(n)}\end{lstlisting}

    \subsection*{pfinite(o, lowerTail, logp)(x)}
    \addcontentsline{toc}{subsection}{pfinite(o, lowerTail, logp)(x)}
    Evaluates the cumulative distribution function at \texttt{x}
for the finite distribution represented by object \texttt{o}:
$$\textrm{pfinite}(o)(x) = \sum\_{k \leq x} dfinite(k)$$


\texttt{lowerTail} defaults to \texttt{true}; if \texttt{lowerTail} is \texttt{false}, returns
the upper tail probability instead:
$$\textrm{pfinite}(o)(x) = \sum\_{k \&gt; x} dfinite(k)$$


\texttt{logp} defaults to \texttt{false}; if \texttt{logp} is \texttt{true}, returns the logarithm
of the result.


    \subsection*{qfinite(o, lowerTail, logp)(p)}
    \addcontentsline{toc}{subsection}{qfinite(o, lowerTail, logp)(p)}
    Evaluates the quantile function for the finite distribution
specified by object \texttt{o}.
In general, for a discrete probability
distribution, the \emph{quantile} is defined as the smallest domain value
\texttt{x} such that $F(x) \geq p$, where $F$ is the cumulative
distribution function.


\texttt{p} is the desired probability ($0 \leq p \leq 1$).


\texttt{lowerTail} defaults to \texttt{true}; if \texttt{lowerTail} is \texttt{false}, \texttt{p} is
interpreted as an upper tail probability.


\texttt{logp} defaults to \texttt{false}; if \texttt{logp} is \texttt{true}, interprets \texttt{p} as
the logarithm of the desired probability.


\texttt{qfinite} tries to invert \texttt{pfinite} but cannot be an exact inverse.
In particular, for \texttt{lowerTail = true}:


\begin{itemize}

\item if asked for the smallest quantile for which the left area (\&lt;=) is 0, \texttt{qfinite} returns
\texttt{min}

\item if asked for the smallest quantile for which the left area (\&lt;=) is 1,
\texttt{qfinite} returns \texttt{max}.

\end{itemize}

The edge cases for \texttt{lowerTail = false} are symmetrical to the preceding:
\texttt{qfinite} returns \texttt{min} or \texttt{max} for a right-tail area (\&gt;) of 1 or 0, respectively.


    \subsection*{rfinite(o)}
    \addcontentsline{toc}{subsection}{rfinite(o)}
    Returns a random variate from the finite distribution
specified by object \texttt{o}.


  \section{gamma}
    \subsection*{dgamma(a, s, logp)(x)}
    \addcontentsline{toc}{subsection}{dgamma(a, s, logp)(x)}
    Evaluates the Gamma density function at \texttt{x}:
$$\textrm{dgamma}(a, s)(x) = \frac{1}{s^a\Gamma(a)}x^{a-1}e^{-x/s}$$


Expects $a \&gt; 0$, $s \&gt; 0$, and $x \&gt; 0$.


\texttt{logp} defaults to \texttt{false}; if \texttt{logp} is \texttt{true}, returns the
logarithm of the result.


    \subsection*{gammadistr(a, s)}
    \addcontentsline{toc}{subsection}{gammadistr(a, s)}
    Returns an object representing a gamma distribution, with properties \texttt{d}, \texttt{p}, \texttt{q}, \texttt{r}.


\begin{lstlisting}
\texttt{gammadistr(a, s).d(x, logp)            // same as dgamma(a, s, logp)(x)
gammadistr(a, s).p(x, lowerTail, logp) // same as pgamma(a, s, lowerTail, logp)(x)
gammadistr(a, s).q(x, lowerTail, logp) // same as qgamma(a, s, lowerTail, logp)(x)
gammadistr(a, s).r()                   // same as rgamma(a, s)()}\end{lstlisting}

    \subsection*{pgamma(a, s, lowerTail, logp)(x)}
    \addcontentsline{toc}{subsection}{pgamma(a, s, lowerTail, logp)(x)}
    Evaluates the Gamma cumulative distribution
function at \texttt{x} (lower tail probability):
$$\textrm{pgamma}(a, s)(x) = \frac{1}{s^a\Gamma(a)}\int\_0^x t^{a-1}e^{-t/s}dt$$


\texttt{lowerTail} defaults to \texttt{true}; if \texttt{lowerTail} is \texttt{false}, returns
the upper tail probability instead.


\texttt{logp} defaults to \texttt{false}; if \texttt{logp} is \texttt{true}, returns the logarithm
of the result.


Expects $a \&gt; 0$, $s \&gt; 0$, and $x \&gt; 0$.


Based on: \emph{Computation of the Incomplete Gamma Function Ratios and their
Inverse}, by DiDonato and Morris, 1992


    \subsection*{qgamma(a, s, lowerTail, logp)(p)}
    \addcontentsline{toc}{subsection}{qgamma(a, s, lowerTail, logp)(p)}
    Evaluates the Gamma distribution's quantile function (inverse cdf) at \texttt{p}:
$$\textrm{qgamma}(a, s)(p) = x \textrm{ such that } \textrm{prob}(X \leq x) = p$$
where $X$ is a random variable with the $\textrm{Gamma}(a, s)$ distribution.


\texttt{lowerTail} defaults to \texttt{true}; if \texttt{lowerTail} is \texttt{false}, \texttt{p} is
interpreted as an upper tail probability (returns
$x$ such that $\textrm{prob}(X \&gt; x) = p)$.


\texttt{logp} defaults to \texttt{false}; if \texttt{logp} is \texttt{true}, interprets \texttt{p} as
the logarithm of the desired probability.


Expects $a\&gt;0$, $s\&gt;0$, and $0 \leq p \leq 1$.


    \subsection*{rgamma(a, s)}
    \addcontentsline{toc}{subsection}{rgamma(a, s)}
    Returns a random variate from the $\textrm{Gamma}(a, s)$ distribution.


Expects $a \&gt; 0$ and $s \&gt; 0$.


TODO: Add links to algorithms


  \section{normal}
    \subsection*{dnorm(mu, sigma, logp)(x)}
    \addcontentsline{toc}{subsection}{dnorm(mu, sigma, logp)(x)}
    Evaluates the Normal density function at \texttt{x}:
$$\textrm{dnorm}(\mu, \sigma)(x) = \frac{1}{\sigma \sqrt{2\pi}} e^{\displaystyle -\frac{(x-\mu)^2}{2\sigma^2}}$$


Expects $\sigma \&gt; 0$.


\texttt{logp} defaults to \texttt{false}; if \texttt{logp} is \texttt{true}, returns the
logarithm of the result.


    \subsection*{normal(mu, sigma)}
    \addcontentsline{toc}{subsection}{normal(mu, sigma)}
    Returns an object representing a normal distribution, with properties \texttt{d}, \texttt{p}, \texttt{q}, \texttt{r}.


\begin{lstlisting}
\texttt{normal(mu, sigma).d(x, logp)            // same as dnorm(mu, sigma, logp)(x)
normal(mu, sigma).p(x, lowerTail, logp) // same as pnorm(mu, sigma, lowerTail, logp)(x)
normal(mu, sigma).q(x, lowerTail, logp) // same as qnorm(mu, sigma, lowerTail, logp)(x)
normal(mu, sigma).r()                   // same as rnorm(mu, sigma)()}\end{lstlisting}

    \subsection*{pnorm(mu, sigma, lowerTail, logp)(x)}
    \addcontentsline{toc}{subsection}{pnorm(mu, sigma, lowerTail, logp)(x)}
    Evaluates the lower-tail cdf at \texttt{x} for the Normal distribution:
$$\textrm{pnorm}(\mu, \sigma)(x) = \frac{1}{\sigma \sqrt{2\pi}}\int\_{-\infty}^x e^{\displaystyle -\frac{(t-\mu)^2}{2\sigma^2}}dt$$


\texttt{lowerTail} defaults to \texttt{true}; if \texttt{lowerTail} is \texttt{false}, returns
the upper tail probability instead.


\texttt{logp} defaults to \texttt{false}; if \texttt{logp} is \texttt{true}, returns the logarithm
of the result.


Expects $\sigma \&gt; 0$.


    \subsection*{qnorm(mu, sigma, lowerTail, logp)(p)}
    \addcontentsline{toc}{subsection}{qnorm(mu, sigma, lowerTail, logp)(p)}
    Evaluates the Normal distribution's quantile function (inverse cdf) at \texttt{p}:
$$\textrm{qnorm}(\mu, \sigma)(p) = x \textrm{ such that } \textrm{prob}(X \leq x) = p$$
where $X$ is a random variable with the $N(\mu,\sigma)$ distribution.


\texttt{lowerTail} defaults to \texttt{true}; if \texttt{lowerTail} is \texttt{false}, \texttt{p} is
interpreted as an upper tail probability (returns
$x$ such that $\textrm{prob}(X \&gt; x) = p)$.


\texttt{logp} defaults to \texttt{false}; if \texttt{logp} is \texttt{true}, interprets \texttt{p} as
the logarithm of the desired probability.


Expects $\sigma \&gt; 0$.


    \subsection*{rnorm(mu, sigma)}
    \addcontentsline{toc}{subsection}{rnorm(mu, sigma)}
    Returns a random variate from the $N(\mu, \sigma)$ distribution.


Expects $\sigma \&gt; 0$.


Uses a rejection polar method.


  \section{poisson}
    \subsection*{dpois(lambda, logp)(x)}
    \addcontentsline{toc}{subsection}{dpois(lambda, logp)(x)}
    Evaluates the Poisson pmf at \texttt{x}:
$$\textrm{dpois}(\lambda)(x) = \frac{\lambda^x}{x!}e^{-\lambda}$$


Expects $\lambda \&gt; 0$.


\texttt{logp} defaults to \texttt{false}; if \texttt{logp} is \texttt{true}, returns the
logarithm of the result.


    \subsection*{pois(lambda)}
    \addcontentsline{toc}{subsection}{pois(lambda)}
    Returns an object representing a Poisson distribution
for a given value of $\lambda \&gt; 0$, with properties \texttt{d}, \texttt{p}, \texttt{q}, \texttt{r}.


\begin{lstlisting}
\texttt{pois(a, b).d(x, logp)            // same as dpois(a, b, logp)(x)
pois(a, b).p(x, lowerTail, logp) // same as ppois(a, b, lowerTail, logp)(x)
pois(a, b).q(x, lowerTail, logp) // same as qpois(a, b, lowerTail, logp)(x)
pois(a, b).r()                   // same as rpois(a, b)()}\end{lstlisting}

    \subsection*{ppois(lambda, lowerTail, logp)(x)}
    \addcontentsline{toc}{subsection}{ppois(lambda, lowerTail, logp)(x)}
    Evaluates the lower-tail cdf at \texttt{x} for the Poisson distribution:
$$\textrm{ppois}(\lambda)(x) = e^{-\lambda} \sum\_{i=0}^{\left \lfloor{x}\right \rfloor}\frac{\lambda^i}{i!}$$


Expects $\lambda \&gt; 0$.


\texttt{lowerTail} defaults to \texttt{true}; if \texttt{lowerTail} is \texttt{false}, returns
the upper tail probability ($P(X \&gt; x)$) instead.


\texttt{logp} defaults to \texttt{false}; if \texttt{logp} is \texttt{true}, returns the logarithm
of the result.


    \subsection*{qpois(lambda, lowerTail, logp)(p)}
    \addcontentsline{toc}{subsection}{qpois(lambda, lowerTail, logp)(p)}
    Evaluates the Poisson distribution's quantile function
(inverse cdf) at \texttt{p}:
$$\textrm{qpois}(\lambda)(p) = x \textrm{ such that } \textrm{prob}(X \leq x) = p$$
where $X$ is a random variable with the $\textrm{Poisson}(\lambda)$ distribution.


Expects $\lambda \&gt; 0$ and $0 \leq p \leq 1$.


\texttt{lowerTail} defaults to \texttt{true}; if \texttt{lowerTail} is \texttt{false}, \texttt{p} is
interpreted as an upper tail probability (returns
$x$ such that $\textrm{prob}(X \&gt; x) = p)$.


\texttt{logp} defaults to \texttt{false}; if \texttt{logp} is \texttt{true}, interprets \texttt{p} as
the logarithm of the desired probability.


    \subsection*{rpois(lambda)}
    \addcontentsline{toc}{subsection}{rpois(lambda)}
    Returns a random variate from the $\textrm{Poisson}(\lambda)$ distribution.


Expects $\lambda \&gt; 0$.


  \section{t}
    \subsection*{dt(df, logp)(x)}
    \addcontentsline{toc}{subsection}{dt(df, logp)(x)}
    Evaluates the t density function at \texttt{x}:
$$\textrm{dt}(\textrm{df})(x) = \frac{1}{\sqrt{\textrm{df}}\,\textrm{B}(1/2, \textrm{df}/2)} \left(1+\frac{x^2}{\textrm{df}} \right)^{-\frac{\textrm{df} + 1}{2}}$$


Expects $\textrm{df} \&gt; 0$.


\texttt{logp} defaults to \texttt{false}; if \texttt{logp} is \texttt{true}, returns the
logarithm of the result.


    \subsection*{pt(df, lowerTail, logp)(x)}
    \addcontentsline{toc}{subsection}{pt(df, lowerTail, logp)(x)}
    Evaluates the lower-tail cdf at \texttt{x} for the t distribution:
$$\textrm{pt}(\textrm{df})(x) = \int\_{-\infty}^{x} \frac{1}{\sqrt{\textrm{df}}\,\textrm{B}(1/2, \textrm{df}/2)} \left(1+\frac{x^2}{\textrm{df}} \right)^{-\frac{\textrm{df} + 1}{2}}$$


Expects $\textrm{df} \&gt; 0$.


\texttt{lowerTail} defaults to \texttt{true}; if \texttt{lowerTail} is \texttt{false}, returns
the upper tail probability instead.


\texttt{logp} defaults to \texttt{false}; if \texttt{logp} is \texttt{true}, returns the logarithm
of the result.


    \subsection*{qt(df, lowerTail, logp)(p)}
    \addcontentsline{toc}{subsection}{qt(df, lowerTail, logp)(p)}
    Evaluates the t distribution's quantile function
(inverse cdf) at \texttt{p}:
$$\textrm{qt}(\textrm{df})(p) = x \textrm{ such that } \textrm{prob}(X \leq x) = p$$
where $X$ is a random variable with the $t(\textrm{df})$ distribution.


Expects $\textrm{df} \&gt; 0$ and $0 \leq p \leq 1$.


\texttt{lowerTail} defaults to \texttt{true}; if \texttt{lowerTail} is \texttt{false}, \texttt{p} is
interpreted as an upper tail probability (returns
$x$ such that $\textrm{prob}(X \&gt; x) = p)$.


\texttt{logp} defaults to \texttt{false}; if \texttt{logp} is \texttt{true}, interprets \texttt{p} as
the logarithm of the desired probability.


Based on: R code


    \subsection*{rt(df)}
    \addcontentsline{toc}{subsection}{rt(df)}
    Returns a random variate from the $t(\textrm{df})$ distribution.


Expects $\textrm{df} \&gt; 0$.


    \subsection*{tdistr(df)}
    \addcontentsline{toc}{subsection}{tdistr(df)}
    Returns an object representing a t distribution with $\textrm{df} \&gt; 0$
degrees of freedom, with properties \texttt{d}, \texttt{p}, \texttt{q}, \texttt{r}.


\begin{lstlisting}
\texttt{tdistr(a, b).d(x, logp)            // same as dt(a, b, logp)(x)
tdistr(a, b).p(x, lowerTail, logp) // same as pt(a, b, lowerTail, logp)(x)
tdistr(a, b).q(x, lowerTail, logp) // same as qt(a, b, lowerTail, logp)(x)
tdistr(a, b).r()                   // same as rt(a, b)()}\end{lstlisting}

  \section{uniform}
    \subsection*{dunif(a, b, logp)(x)}
    \addcontentsline{toc}{subsection}{dunif(a, b, logp)(x)}
    Evaluates the Uniform density function at \texttt{x}:
$$\textrm{dunif}(a, b)(x) = \begin{cases}
  \frac{1}{b-a},  \&amp; \text{if $a \leq x \leq b$} \\
  0, \&amp; \text{if $x \&lt; a$ or $x \&gt; b$} \end{cases} $$


Expects $a \&lt; b$.


\texttt{logp} defaults to \texttt{false}; if \texttt{logp} is \texttt{true}, returns the
logarithm of the result.


    \subsection*{punif(a, b, lowerTail, logp)(x)}
    \addcontentsline{toc}{subsection}{punif(a, b, lowerTail, logp)(x)}
    Evaluates the lower-tail cdf at \texttt{x} for the Uniform distribution:
$$\textrm{punif}(a, b)(x) = \begin{cases}
  \frac{x-a}{b-a},  \&amp; \text{if $a \leq x \leq b$} \\
  0,                \&amp; \text{if $x \&lt; a$} \\
  1,                \&amp; \text{if $x \&gt; b$} \end{cases} $$


Expects $a \&lt; b$.


\texttt{lowerTail} defaults to \texttt{true}; if \texttt{lowerTail} is \texttt{false}, returns
the upper tail probability instead.


\texttt{logp} defaults to \texttt{false}; if \texttt{logp} is \texttt{true}, returns the logarithm
of the result.


    \subsection*{qunif(a, b, lowerTail, logp)(p)}
    \addcontentsline{toc}{subsection}{qunif(a, b, lowerTail, logp)(p)}
    Evaluates the Uniform distribution's quantile function
(inverse cdf) at \texttt{p}:
$$\textrm{qunif}(a, b)(p) = x \textrm{ such that } \textrm{prob}(X \leq x) = p$$
where $X$ is a random variable with the $\textrm{Uniform}(a, b)$ distribution.


Expects $a \&lt; b$ and $0 \leq p \leq 1$.


\texttt{lowerTail} defaults to \texttt{true}; if \texttt{lowerTail} is \texttt{false}, \texttt{p} is
interpreted as an upper tail probability (returns
$x$ such that $\textrm{prob}(X \&gt; x) = p)$.


\texttt{logp} defaults to \texttt{false}; if \texttt{logp} is \texttt{true}, interprets \texttt{p} as
the logarithm of the desired probability.


    \subsection*{runif(a, b)}
    \addcontentsline{toc}{subsection}{runif(a, b)}
    Returns a random variate from the $\textrm{Uniform}(a,b)$ distribution.


Expects $a\&lt;b$.


    \subsection*{unif(a, b)}
    \addcontentsline{toc}{subsection}{unif(a, b)}
    Returns an object representing a Uniform distribution on $[a, b]$,
with properties \texttt{d}, \texttt{p}, \texttt{q}, \texttt{r}.


\begin{lstlisting}
\texttt{unif(a, b).d(x, logp)            // same as dunif(a, b, logp)(x)
unif(a, b).p(x, lowerTail, logp) // same as punif(a, b, lowerTail, logp)(x)
unif(a, b).q(x, lowerTail, logp) // same as qunif(a, b, lowerTail, logp)(x)
unif(a, b).r()                   // same as runif(a, b)()}\end{lstlisting}

  \section{rgen}
    \subsection*{getAlgorithms()}
    \addcontentsline{toc}{subsection}{getAlgorithms()}
    Returns a list of the available algorithms for random number
generation.


    \subsection*{random()}
    \addcontentsline{toc}{subsection}{random()}
    Generates a random number uniformly on the interval $(0, 1)$.


    \subsection*{setAlgorithm(name)}
    \addcontentsline{toc}{subsection}{setAlgorithm(name)}
    Sets the algorithm to be used for random number generation.
Expects \texttt{name} to be one of the strings returned by \texttt{getAlgorithms}.


    \subsection*{setRandomSeed()}
    \addcontentsline{toc}{subsection}{setRandomSeed()}
    Sets a seed for random number generation, based on the system date.


    \subsection*{setSeed(seed)}
    \addcontentsline{toc}{subsection}{setSeed(seed)}
    Sets a seed for the random number generator.  Expects \texttt{seed} to
be an integer.


\end{document}
